Cloud computing is the concept of computing remotely through Internet on data centers where thousands of computers/server and other communication devices are installed and running 24/7. Now-a-days businesses are moving toward cloud for their need for computing and storage to take the advantage of economy of scale model of cloud computing. Major services are IaaS – Infrastructure as Service, PaaS - Platform as Service, SaaS – Software as Service, all are unique solutions targeted to meet diverse customers needs. Cloud service providers provide services to their clients by sharing available physical resources~\cite{5331755} at data center with clients. There are three kinds of resources namely: disk storage, communication bandwidth, and processing capacity; these resources are shared/rented with clients according to agreed service-level-agreement (SLA). Furthermore, the allocation of resources such as communication and computation resources is done to maximize the profit of cloud service providers while meeting the contract SLA~\cite{5176040,anand2013virtual}.

When customers leave a service provider for another service provider they are known as churners and the phenomena is called customer churn. While customer acquisition is good for the business, customer churn is an important issue that threatens the stability of cloud computing businesses. Churn results into losing customer to another business provider which causes serious revenue loss ~\cite{Huang:2012:CCP:2038068.2038213, cloudchurn2012}. There are two types of customer churn, voluntary and involuntary. A churn is voluntary when a customer decides to leave a service provider because of unsatisfactory service quality, pricing, billing etc. Poor customer service is also a major factor that leads to customer churn.  Involuntary churn is when customers have to leave a service provider in the events that are beyond the customer's control such as death, geographic relocation etc.  Involuntary churn cannot be controlled, however, voluntary churn can be addressed by businesses by providing various incentives to users to stay with the business. 


Like any other industry, cloud computing industry is no stranger to the issue of customer churn since it also competes with other cloud computing businesses for customers.  It is reported that even with 2\% monthly churn, more than half of cloud customers could be gone in three years~\cite{cloudchurn2012}. Large churn rate can result into severe financial consequences that may lead to downsizing of the businesses. Furthermore, such instability in cloud computing businesses can result into sudden job losses and wastage of resources that were bought and would not be used. While 100\% customer retention is an unrealistic goal the customer churn can be reduced or further be controlled by providing incentives to users in the form of better service and price combinations.  Service providers can effectively allocate resources by finding a tradeoff between profit and customer satisfaction. Improvement in customer satisfaction increases the customer loyalty. Besides it is easy to retain a customer than acquire a new one. Strategies that take user satisfaction of churners into account to provide more or better services for the same or lower price are proven to be more effective~\cite{Burez2007277,6932875,TamaddoniJahromi20141258} in retaining customers than methods based on managerial heuristics. Our work utilizes this strategy to address customer churn problem in cloud computing service industry.  

%Cloud computing is a kind of computing based on Internet that combine Distributed Computing, Parallel Computing, and Grid Computing etc. In cloud computing, customers could pay for different types of Virtual Machines(VMs)on their demand and receive resources from cloud service providers after they send requests. Companies provide online services such as software applications, hardware, and data storage to different customers. There are several services that cloud computing providers provide to satisfy the different business -- SaaS (Software as a Service), PaaS (Platform as a Service), and IaaS (Infrastructure as a Service).


We propose a novel profit maximizing retention framework that integrates customer experience with resource allocation and placement. In such a context, following are the contributions of the paper:
\begin{itemize}
	\item We propose a churn aware resource allocation and virtual machine placement framework.
\item The paper proposes two types of retention action policies.
	\item Possible method for churn classification and churn degree ( churn propensity) computation is suggested.
	\item Model based approach for virtual machine (VM) overheads accounting for operation overhead and interference overhead is proposed.
	\item Virtual machine placement problem accounting for two types of overheads is formulated as integer programming problem. 
	\item Experimental result utilizes customer data inspired by real data set from telecommunication industry.
\end{itemize}

The rest of the paper is organized as follows: Section \ref{related} briefly describes background and related work on this topic. Cloud system models are introduced and described in Section~\ref{model}. Section~\ref{retention} describes our method for resource allocation and VM placement. Experimental result is presented in Section~\ref{experiment}. Finally, summary and future work are described in Section~\ref{future}.


%In the cloud computing system, there are three different parties including infrastructure service providers, business service providers and customers (end users)\cite{chen2011tradeoffs}.  Business service provider rent VM (Virtual Machine) instances from infrastructure service providers to provider resources to customers. The profit for a business service provider depend on payment of each customer and the cost of charging to infrastructure service provider to satisfy the customer service requirement. All customers will consider the price for different type of VM instances and response time. They will leave to another business service provider if these two conditions cannot  asatisfy customers. Losing customers is a serious damage for a business, and this phenomena is known as churn. Even though more and more people pay their attention to cloud computing, cloud computing businesses are facing customer churn problem like other businesses.

%The Quality of Service is significate factor for a successful business, and customers determine if they stay with a same service provider depend on it. When customers feel unsatisfied with the quality of service for a company, such as high price, they can change their mind and switch to another company. There are also some churners who terminate their contract without switching to another company, but financial issues or other problems. These two types of churners are called as voluntary churner. According to different reason of switching, churners can divide by three types: active,rotational,and passive\cite{lazarov2007churn}. It is vital that prevent customers to quit contract, so companies have to know whom are leaving and why they are leaving.
