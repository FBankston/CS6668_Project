This paper presents a customer churn aware resource allocation and virtual machine placement framework for cloud. The proposed idea rests on a successful strategy of improving customer satisfaction levels by providing better/more services. 
The proposed framework utilizes historical customer knowledge base to identify possible churners and under a retention action policy it allocates additional resources to the dissatisfied customers. Allocated resources in the form of VM is placed on PM using a profit maximization framework. The formulated profit maximization framework uses operation as well as interference overheads. Additionally, we suggest a machine learning framework to identify churners. We show the effectiveness of our work through extensive simulation by using real life as well as synthetic data. 

This is the first work that proposes customer churn awareness in resource allocation and placement in cloud service industry. Although the processor speed is used as a capacity parameters in our work the framework is equally applicable when definition of capacity is expanded to include other resources such as storage and memory. Another limiting factor of our work is the retention action where it is not certain whether the offered retention action would be accepted by the user, therefore, our future work aims to include this consideration to offer a meaningful retention action for cloud users. Our proposed work is based on strategy that better experience/service means better satisfaction, however, estimation of improvement in satisfaction level under this strategy is better suited for cloud service providers that we leave for future . Although issue of customer churn can be addressed by considering several other vital factors in cloud computing service industry our simple proposal is an attempt to promote the research and development effort in this direction.  


%In this paper, we talk both churn rate and VMs placement, our aim to reduce the churn rate to a safe line, because high churn rate is harmful for a company or a service provider. There are many ways to reduce the churn rate, and Retention Action is a common way. From the above sections, we knew that how much Retention Action the service provider need to give to the customers under its real situation based on function $f(r_c)$. This assumed function may not accurate enough or the service provider has its own model. For this issue, we still need some more big data to test and verify it, and then we can come up with an accurate function model. Furthermore, there are two schemes we talked in section IV, but scheme 2 (Ambulatory Retention Action) is flexible, the service provider doesn’t have a budget on Retention Action, so the churn rate could be reduced more than fixed scheme, in this case, service provider can use least retention action to get a lowest churn rate and highest total profit. This paper addressed how to reduce the churn rate in cloud service, and how to place VMs for decreasing cost. Giving the retention action to customers is a good way to enhance customers’ degree of satisfaction. On the other hand, an excellent placement strategy can dramatically decrease cost of PMs, so we can get an optimal profit.